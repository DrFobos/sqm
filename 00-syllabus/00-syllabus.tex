% (The MIT License)
%
% Copyright (c) 2023-2024 Yegor Bugayenko
%
% Permission is hereby granted, free of charge, to any person obtaining a copy
% of this software and associated documentation files (the 'Software'), to deal
% in the Software without restriction, including without limitation the rights
% to use, copy, modify, merge, publish, distribute, sublicense, and/or sell
% copies of the Software, and to permit persons to whom the Software is
% furnished to do so, subject to the following conditions:
%
% The above copyright notice and this permission notice shall be included in all
% copies or substantial portions of the Software.
%
% THE SOFTWARE IS PROVIDED 'AS IS', WITHOUT WARRANTY OF ANY KIND, EXPRESS OR
% IMPLIED, INCLUDING BUT NOT LIMITED TO THE WARRANTIES OF MERCHANTABILITY,
% FITNESS FOR A PARTICULAR PURPOSE AND NONINFRINGEMENT. IN NO EVENT SHALL THE
% AUTHORS OR COPYRIGHT HOLDERS BE LIABLE FOR ANY CLAIM, DAMAGES OR OTHER
% LIABILITY, WHETHER IN AN ACTION OF CONTRACT, TORT OR OTHERWISE, ARISING FROM,
% OUT OF OR IN CONNECTION WITH THE SOFTWARE OR THE USE OR OTHER DEALINGS IN THE
% SOFTWARE.

\documentclass[nobrand,anonymous,nodate,nosecurity]{huawei}
\def\SqmEnglish{}
\ifdefined\SqmRussian
  \usepackage[utf8]{inputenc}
  \usepackage[T2A]{fontenc}
  \usepackage[russian,english]{babel}
\else
  \usepackage[utf8]{inputenc}
  \usepackage[T1]{fontenc}
  \usepackage[english]{babel}
\fi
\usepackage{booktabs}
\usepackage{wrapfig}
\usepackage{tabularx}
\usepackage{iexec}
\usepackage{multicol}
\usepackage{href-ul}
\usepackage{ffcode}
\usepackage{soul}
\begin{document}

\newcommand\angry[1]{\textcolor{red}{#1}}

\iexec[quiet]{wget -N --quiet https://www.yegor256.com/images/face-1024x1024.jpg}

\ifdefined\SqmEnglish
{\sffamily{\bfseries\Large Software Quality Metrics (SQM)}

\begin{abstract}
In the course, students will learn different approaches to software quality,
starting from the most traditional ones, like McCabe's complexity and
Halstead effort, and finishing with most recently developed, based on Neural Networks
and Large Language Models (LLM).
\end{abstract}

\textbf{What is the goal?}\\
Writing code that compiles is easy. Writing code that is easy to read
is a much bigger challenge. Without a proper understanding of what maintainability
and software quality mean and why they matter, it may be hard or even impossible
to become a professional programmer, capable of designing large software systems.
At this course, we will attempt to investigate this very subject
in order to increase the level of students' professionalism.

\floatstyle{plain}
\restylefloat{figure}
\begin{wrapfigure}{r}{1.2in}%
\raggedleft%
\includegraphics[width=1.2in]{face-1024x1024.jpg}%
\end{wrapfigure}
\textbf{Who is the teacher?}\\
Yegor is developing software for more than 30 years, being a hands-on programmer
(see his GitHub account with 4.9K followers: \href{https://github.com/yegor256}{@yegor256})
and a manager of other programmers. At the moment, he is a director
of an R\&D laboratory in Huawei. His recent conference talks are in
\href{https://www.youtube.com/channel/UCr9qCdqXLm2SU0BIs6d_68Q}{his YouTube channel}.
He also published a \href{https://www.yegor256.com/books.html}{few books}
and wrote a \href{https://www.yegor256.com/contents.html}{blog} about software engineering
and object-oriented programming.
He previously taught a few courses in
\href{https://innopolis.university/}{Innopolis University} (Kazan, Russia)
and
\href{https://hse.ru}{HSE University} (Moscow, Russia),
for example,
\href{https://github.com/yegor256/ssd16}{SSD16 (2021)},
\href{https://github.com/yegor256/eqsp}{EQSP (2022)},
\href{https://github.com/yegor256/ppa}{PPA (2023)},
\href{https://github.com/yegor256/painofoop}{COOP (2023)},
and
\href{https://github.com/yegor256/pmba}{PMBA (2023)}
(all videos are available).

\textbf{Why this course?}\\
Unfortunately, the quality of software is not getting better in the
industry of software development while the amount of code we programmers
write every year steadily grows. There is an urgent need to help
programmers understand their craft better through the perspective
of quantitative and qualitative analysis of software code.

\textbf{What's the methodology?}\\
There are 24 lectures, each dedicated to one (or a few) metrics that
were designed to measure the quality or quantity of code over the last
few decades of software engineering and computer science. Students will
hear the theories behind the metrics accompanied with the stories
experienced by the lecturer in his practical participation in software
projects.

\newpage
\section*{Course Structure}

Prerequisites to the course (it is expected that a student knows this):

\begin{itemize}
\item How to write code
\item How to design software
\item How to use Git
\end{itemize}

After the course, students \emph{hopefully} will understand:

\begin{itemize}
\item How to make their bug reports appreciated?
\item How to make their pull requests merged?
\item How to reject a pull request politely?
\item How to become an active contributor of a large repository?
\item How to keep up with GitHub etiquette?
\item How to invite and motivate contributors?
\item How to deal with frustration during code reviews?
\item How to avoid stale pull requests (never merged)?
\item How to use GitHub Actions effectively?
\item How to format the \texttt{README.md} file?
\item How to control quality and avoid chaos in a public repository?
\item How to use GitHub account in lieu of a C.V.?
\item How to get 100 stars?
\item How to release in one click?
\item How to employ ChatGPT as a coding companion?
\item How to get 10K stars?
\item How to earn money via open source?
\end{itemize}

\newpage
\section*{Lectures \& Labs}

The following 80-minute lectures constitute the course:

\newlist{lectures}{enumerate}{10}
\setlist[lectures]{label*=\arabic*.}
\begin{lectures}
    \item Lines of Code
    \item McCabe's Cyclomatic Complexity
    \item Cognitive Complexity
    \item Halstead Complexity
    \item Maintainability Index
    \item Coupling
    \item LCOM 1, 2, 3, 4, 5
    \item TCC and LCC
    \item CAMC and NHD
    \item Object Dimensions
    \item Clone Coverage
    \item Dead Code
    \item Code Churn
    \item Tech Debt
    \item Code Coverage
    \item Mutation Coverage
    \item Function Points
    \item Defects Density
    \item Comments Density
    \item Commits Density
    \item Builds
    \item Code Style
    \item Static Analysis
    \item Neural Metrics
\end{lectures}

At laboratory classes, organized by a Teaching Assistant (TA),
students either make pull requests to
GitHub repositories suggested by the teacher, or write sections
for their research papers.

Most probably, one of the following repositories will be suggested
by the teacher for contribution during the course:
\href{https://github.com/yegor256/cam}{yegor256/cam} (Bash, Python),
\href{https://github.com/yegor256/rultor}{yegor256/rultor} (Java, XML),
\href{https://github.com/yegor256/qulice}{yegor256/qulice} (Java),
\href{https://github.com/cqfn/jpeek}{cqfn/jpeek} (Java, XML),
and
\href{https://github.com/objectionary/eo}{objectionary/eo} (Java, XSLT).

\newpage
\section*{Grading}

At the first lecture, students form groups of \textbf{2--3} people in each one (no exceptions!).
Each group picks a research topic from the list suggested by the teacher.

Each group writes a research paper in \LaTeX, according
to the \href{https://www.yegor256.com/2022/08/24/research-paper-template.html}{guideline}.
The length of the pager may not exceed four pages in
\href{https://ctan.org/pkg/acmart}{acmart/sigplan} 10pt format
(including references and appendices).
The paper must be presented to the teacher \angry{incrementally}, section by section:
\begin{inparaenum}[1)]
\item Method,
\item Related Work,
\item Results,
\item Discussion,
\item Conclusion,
\item Introduction,
and
\item Abstract.
\end{inparaenum}
Once a section is \angry{approved} by the teacher, the next section may be presented for review.

After a presentation of a section, the teacher may ask the group to \angry{stop}
working with the paper. In this case, no sections may be presented for review any more: they all will be rejected.
This decision is \angry{subjectively} made by the teacher and will \angry{not} be explained
to the students, however the following may contribute to such a
negative decision:
\begin{inparaenum}[a)]
    \item ChatGPT,
    \item plagiarism,
    \item negligence,
    and
    \item laziness.
\end{inparaenum}

When the Abstract is accepted by the teacher, a group may ask a student from another
group to review their paper, according to
\href{https://www.yegor256.com/2023/12/17/how-to-review-research-paper.html}{this guideline}.
The review must be accepted by the TA.

There is no exam at the end of the course. Instead,
each student earns points for the following results:\\
\renewcommand{\arraystretch}{1}
\begin{tabular}{lrr}
Result & Points & Limit \\
\hline
Attended a lecture & +2 & 8 \\
Attended a lab & +2 & 8 \\
Merged a pull request to a suggested repo & +4 & 48 \\
Reviewed a paper of another group & +3 & 9 \\
``Method'' section (\href{https://www.yegor256.com/2023/10/11/method-of-research.html}{guideline}) & +8 \\
``Related Work'' section (\href{https://www.yegor256.com/2023/09/29/how-to-write-related-work-section.html}{guideline}) & +12 \\
``Results'' and ``Discussion'' sections (\href{https://www.yegor256.com/2023/12/11/results-and-discussion.html}{guideline}) & +12 \\
``Conclusion'' & +6 \\
``Introduction'' section & +6 \\
``Abstract'' and Title & +4 \\
\end{tabular}

Then, 55+ points mean ``A,'' 47+ mean ``B,'' and 23+ mean ``C.''

An online lecture is counted as ``attended'' only if a student was personally
presented in Zoom for more than 75\% of the lecture's time. Watching the
lecture from the computer of a friend doesn't count.

\else

\newcommand\head[2]{#1: & #2 \\}
\begin{tabularx}{\textwidth}{r>{\raggedright\arraybackslash}X}
\head{Название}
  {``Практические методы повышения качества программного кода''}
\head{Департамент}
  {Кафедра Программной Инженерии (КПИ)}
\head{Период реализации}
  {1,2,3 модули 2023/2024}
\head{Язык}
  {Русский}
\head{Видео}
  {\href{https://www.youtube.com/playlist?list=PLaIsQH4uc08xyXRhhYPHh-Yam2kEwNaLl}{YouTube}}
\head{Слайды}
  {\href{https://github.com/yegor256/sqm}{GitHub}}
\head{Охват аудитории}
  {Для всего кампуса}
\head{Объем дисциплины}
  {48 часов семинары + 66 часов самостоятельная работа}
\head{Онлайн курс}
  {---}
\head{Технологии реализации}
  {Лекции: офлайн-занятия, практика: офлайн-занятия}
\head{Разработчики}
  {Бугаенко Егор Георгиевич}
\head{Утверждение}
  {---}
\end{tabularx}

\section{Цели курса}

В лекционной части курса рассматриваются инструменты оценки качества кода и методы его повышения. Теоретическое изложение не привязано к конкретным языкам программирования. Практическая часть курса предполагает самостоятельную работу студентов над анализом крупных open source проектов с целью выявления закономерностей и проблем с качеством.

Ожидается, что по окончанию курса студент будет:
\begin{itemize}
    \item Понимать, из чего состоит оценка качества программных продуктов;
    \item Уметь проводить научные исследования в сфере анализа репозиториев с программным кодом;
    \item Уметь пользоваться инструментами анализа качества кода.
\end{itemize}

Предполагается, что к началу курса студент умеет:
\begin{itemize}
    \item Самостоятельно программировать на одном из популярных языков, таких как Java, JavaScript, Python или C++;
    \item Самостоятельно управлять репозиторием проекта с помощью Git;
    \item Самостоятельно моделировать архитектуру программного проекта, используя UML.
\end{itemize}

Лучшие студенты опубликуют результаты своего исследования на одной из научных конференций.

\section{Лекции}

Всего в курсе 24 лекции по два академических часа (80 минут каждая лекция):
\begin{enumerate}
    \setlength\itemsep{0em}
    \item Lines of Code
    \item McCabe's Cyclomatic Complexity
    \item Cognitive Complexity
    \item Halstead Complexity
    \item Maintainability Index
    \item Coupling
    \item LCOM 1, 2, 3, 4, 5
    \item TCC and LCC
    \item CAMC and NHD
    \item Object Dimensions
    \item Clone Coverage
    \item Dead Code
    \item Code Churn
    \item Tech Debt
    \item Code Coverage
    \item Mutation Coverage
    \item Function Points
    \item Defects Density
    \item Comments Density
    \item Commits Density
    \item Builds
    \item Code Style
    \item Static Analysis
    \item Neural Metrics
\end{enumerate}

Порядок лекций может меняться, а тематика корректироваться в зависимости от интереса аудитории.

Тип аудиторий: лекционные (для лекционных занятий).

Оснащение аудиторий: наличие Wi-Fi.

\section{Аттестация}

В рамках курса студенты должны разделиться на группы по 1--3 человека. Каждая группа должна выбрать одну из исследовательских тем, предложенных на первой лекции. По выбранной теме необходимо провести исследование и оформить его результаты в научную статью на английском языке, объемом 7--12 страниц формата \href{https://ctan.org/pkg/acmart}{acmart/sigplan}. Статья должна содержать обязательные разделы: Abstract, Introduction, Related Work, Method, Experimental Results, Discussion, Conclusion и References.

Студент может также набрать баллы путем написания кода в одном из open source репозитории.

Оценка за курс складывается из суммы баллов:

\renewcommand{\arraystretch}{1}
\begin{tabularx}{\textwidth}{>{\raggedright}p{4in}>{\raggedleft\arraybackslash}X}
\toprule
Результат & Баллы \\
\midrule
Студент представил черновик статьи (2+ страниц) до конца первого модуля & 1 \\
Студент представил черновик статьи (4+ страниц) до конца второго модуля & 1 \\
Студент успешно отправил 4+ pull requests & 1 \\
Студент успешно отправил 8+ pull requests & 2 \\
Студент успешно отправил 12+ pull requests & 4 \\
Студент посетил 10 или более лекций & 1 \\
Студент посетил 20 или более лекций & 3 \\
Студент отправил статью на конференцию после разрешения преподавателя & 1 \\
Статья принята на конференции CORE-B & 7 \\
Статья принята на конференции CORE-A & 10 \\
\bottomrule
\end{tabularx}

Пересдача невозможна.

\section{Самостоятельная работа}

Работая самостоятельно над темой исследования, ожидается, что студент выполнит следующее:
\begin{itemize}
    \item Изучит существующие статьи и книги, касающиеся темы исследования;
    \item Сформулирует research question;
    \item Определит метод исследования;
    \item Проведет исследования и соберет данные;
    \item Проанализирует результаты;
    \item Выявит слабые стороны и опишет их;
    \item Сделает вывод;
    \item Оформит статью.
\end{itemize}

Дополнительную помощь можно получить в этой статье:
\href{https://www.yegor256.com/2022/08/24/research-paper-template.html}{Research Paper Simple Template}
(и в материалах, на которые статья ссылается).

\section{Литература}

\begin{multicols}{2}\small\raggedright
Len Bass et al., \ul{Software Architecture in Practice}\\[3pt]
Paul Clements et al., \ul{Documenting Software Architectures: Views and Beyond}\\[3pt]
\nospell{Karl Wiegers} et al., \ul{Software Requirements}\\[3pt]
{\nospell{Alistair Cockburn}}, \ul{Writing Effective Use Cases}\\[3pt]
{\nospell{Steve McConnell}}, \ul{Software Estimation: Demystifying the Black Art}\\[3pt]
{Robert Martin}, \ul{Clean Architecture: A Craftsman's Guide to Software Structure and Design}\\[3pt]
{Steve McConnell}, \ul{Code Complete}\\[3pt]
{Frederick Brooks Jr.}, \ul{Mythical Man-Month, The: Essays on Software Engineering}\\[3pt]
{David Thomas et al.}, \ul{The Pragmatic Programmer: Your Journey To Mastery}\\[3pt]
{Robert C. Martin}, \ul{Clean Code: A Handbook of Agile Software Craftsmanship}\\[3pt]
{\nospell{Grady Booch} et al.}, \ul{Object-Oriented Analysis and Design with Applications}\\[3pt]
{\nospell{Bjarne Stroustrup}}, \ul{Programming: Principles and Practice Using C++}\\[3pt]
{\nospell{Brett McLaughlin} et al.}, \ul{Head First Object-Oriented Analysis and Design: A Brain Friendly Guide to OOA\&D}\\[3pt]
{David West}, \ul{Object Thinking}\\[3pt]
{Eric Evans}, \ul{Domain-Driven Design: Tackling Complexity in the Heart of Software}\\[3pt]
{Yegor Bugayenko}, \ul{Code Ahead}\\[3pt]
{Yegor Bugayenko}, \ul{Elegant Objects}\\[3pt]
{Michael Feathers}, \ul{Working Effectively with Legacy Code}\\[3pt]
{Martin Fowler}, \ul{Refactoring: Improving the Design of Existing Code}\\[3pt]
{Erich Gamma et al.}, \ul{Design Patterns: Elements of Reusable Object-Oriented Software}\\[3pt]
{Martin Fowler}, \ul{UML Distilled}\\[3pt]
{\nospell{Anneke Kleppe} et al.}, \ul{MDA Explained: The Model Driven Architecture: Practice and Promise}\\[3pt]
{\nospell{Jez Humble} et al.}, \ul{Continuous Delivery: Reliable Software Releases through Build, Test, and Deployment Automation}\\[3pt]
{\nospell{Michael T. Nygard}}, \ul{Release It!: Design and Deploy Production-Ready Software}\\[3pt]
\end{multicols}

\fi

\end{document}
