% (The MIT License)
%
% Copyright (c) 2023-2024 Yegor Bugayenko
%
% Permission is hereby granted, free of charge, to any person obtaining a copy
% of this software and associated documentation files (the 'Software'), to deal
% in the Software without restriction, including without limitation the rights
% to use, copy, modify, merge, publish, distribute, sublicense, and/or sell
% copies of the Software, and to permit persons to whom the Software is
% furnished to do so, subject to the following conditions:
%
% The above copyright notice and this permission notice shall be included in all
% copies or substantial portions of the Software.
%
% THE SOFTWARE IS PROVIDED 'AS IS', WITHOUT WARRANTY OF ANY KIND, EXPRESS OR
% IMPLIED, INCLUDING BUT NOT LIMITED TO THE WARRANTIES OF MERCHANTABILITY,
% FITNESS FOR A PARTICULAR PURPOSE AND NONINFRINGEMENT. IN NO EVENT SHALL THE
% AUTHORS OR COPYRIGHT HOLDERS BE LIABLE FOR ANY CLAIM, DAMAGES OR OTHER
% LIABILITY, WHETHER IN AN ACTION OF CONTRACT, TORT OR OTHERWISE, ARISING FROM,
% OUT OF OR IN CONNECTION WITH THE SOFTWARE OR THE USE OR OTHER DEALINGS IN THE
% SOFTWARE.

\documentclass{article}
\usepackage{../sqm}
\newcommand*\thetitle{Lines of Code (LoC)}
\begin{document}

\plush{\sqmTitlePage{1}{q9Gr2xguP5I}}

\thought{Everybody wants higher \ul{quality of code}, but nobody knows how to measure it.}

\thought{Code \ul{maintainability}, probably, is the ultimate objective of increasing the quality of source code.}

\thought{Code \ul{size} makes the biggest negative impact on code maintainability.}

\thought{It \ul{may} be wrong to measure productivity of a programmer by counting lines of code, but for the quality of code the LoC metric is a perfect indicator.}

\thought{Instead of counting lines, it may be more reasonable to count NCSS (Non Commenting Source Statements), but not always.}

\thought{There are \href{https://www.linux.com/news/linux-in-2020-27-8-million-lines-of-code-in-the-kernel-1-3-million-in-systemd/}{27.8M} lines of C code in Linux kernel. What does it tell us?}

\thought{In 2011, Uncle Bob \href{https://softwareengineering.stackexchange.com/questions/66523}{suggested} that 200 lines per Java class is a good guideline to stay below.}

\thought{Java is \href{http://jameshfisher.github.io/languageredundancy/}{two times} more verbose than Ruby. Does it mean the quality of an average Ruby code is higher?}

\plush{
  \pptBanner{Read this:}\par
  \href{https://www.yegor256.com/2017/12/26/software-quality-formula.html}{The Formula for Software Quality} (2017)\par
  \href{https://www.yegor256.com/2014/04/11/cost-of-loc.html}{How Much Do You Pay Per Line of Code?} (2014)\par
  \href{https://www.yegor256.com/2014/11/14/hits-of-code.html}{Hits-of-Code Instead of SLoC} (2014)\par
  \href{https://www.yegor256.com/2014/08/13/strict-code-quality-control.html}{Strict Control of Java Code Quality} (2014)\par
}

\end{document}
