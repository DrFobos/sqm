% (The MIT License)
%
% Copyright (c) 2023-2024 Yegor Bugayenko
%
% Permission is hereby granted, free of charge, to any person obtaining a copy
% of this software and associated documentation files (the 'Software'), to deal
% in the Software without restriction, including without limitation the rights
% to use, copy, modify, merge, publish, distribute, sublicense, and/or sell
% copies of the Software, and to permit persons to whom the Software is
% furnished to do so, subject to the following conditions:
%
% The above copyright notice and this permission notice shall be included in all
% copies or substantial portions of the Software.
%
% THE SOFTWARE IS PROVIDED 'AS IS', WITHOUT WARRANTY OF ANY KIND, EXPRESS OR
% IMPLIED, INCLUDING BUT NOT LIMITED TO THE WARRANTIES OF MERCHANTABILITY,
% FITNESS FOR A PARTICULAR PURPOSE AND NONINFRINGEMENT. IN NO EVENT SHALL THE
% AUTHORS OR COPYRIGHT HOLDERS BE LIABLE FOR ANY CLAIM, DAMAGES OR OTHER
% LIABILITY, WHETHER IN AN ACTION OF CONTRACT, TORT OR OTHERWISE, ARISING FROM,
% OUT OF OR IN CONNECTION WITH THE SOFTWARE OR THE USE OR OTHER DEALINGS IN THE
% SOFTWARE.

\documentclass{article}
\usepackage{../sqm}
\newcommand*\thetitle{Maintainability Index}
\begin{document}

\plush{\sqmTitlePage{5}{xnSnPfVkmkM}}

\plush{\pptPic{.6}{wtfs.png}}

\qte
  [Fred Brooks]
  {fred-brooks.jpg}
  {The total \ul{cost of maintaining} a widely used program is typically 40 percent or more of the cost of developing it}
  {brooks1982}

\qte
  [Ward Cunningham]
  {ward-cunningham.jpg}
  {Shipping first time code is like going into \ul{debt}. A little debt speeds development so long as it is paid back promptly with a rewrite. The danger occurs when the debt is not repaid. Every minute spent on not-quite-right code counts as interest on that debt.}
  {cunningham1992}

\qte
  [Don Coleman and Dan Ash (Hewlett Packard), Bruce Lowther (Micron Semiconductor), \ul{Paul Oman} (University of Idaho)]
  {anonymous-2.png}
  {Before developers can claim that they are building maintainable systems, there must be some way to \ul{measure} maintainability}
  {coleman1994}

\qte
  {paper-1.png}
  {The factors of software that determine or influence maintainability can be organized into a \ul{hierarchical} structure of measurable attribute. Our hierarchy serves as a taxonomic definition for software maintainability that is compatible with the 35 published works upon which it is based.}
  {oman1992metrics}

\plush{
\pptBanner{Software Maintainability Taxonomy}
\pptPic{.6}{tree.png}\par
{\scriptsize Source: \bibentry{oman1992metrics}\par}}

\plush{
\pptBanner{Maintainability Formula}
\begin{multicols}{2}
\pptPic{.8}{formula-1.png}\par
{\scriptsize Source: \bibentry{oman1992metrics}\par}
\par\columnbreak\par
``This formula represents the product of the weighted dimensions, where each dimension is measured as the average deviation from a known value of 'goodness' for that maintainability attribute.''
\end{multicols}
}

\qte
  [Don Coleman, Dan Ash, Bruce Lowther, Paul Oman]
  {paper-2.png}
  {A software maintainability model is only useful if it can provide developers and maintainers in an industrial setting with more information about the system}
  {coleman1994}

\plush{
\pptBanner{First Approximation}
\begin{multicols}{2}
\pptPic{.95}{formula-2.png}
\par\columnbreak\par
``Approximately 50 regression models were constructed in an attempt to identify simple models that could be calculated from existing tools and still be generic enough to apply to a wide range of software systems. The regression model that seemed most applicable was a four-metric polynomial based on  \begin{inparaenum}[1)]\item Halstead’s effort, \item extended cyclomatic complexity, \item lines of code, and \item number of comments.\end{inparaenum}''
\end{multicols}
}

\plush{
\pptBanner{The Formula of Maintainability Index}
\begin{multicols}{2}
\pptPic{.95}{formula-3.png}
\par\columnbreak\par
\textit{aveVol} --- average Halstead Volume in a module\par
\textit{ave V(g')} --- average total cyclomatic complexity in a module\par
\textit{aveLOC} --- average lines of code in a module\par
\textit{perCM} --- average percent of comments in a module\par
\end{multicols}
}

\plush{
\pptBanner{Maintainability Index by Visual Studio}
\begin{multicols}{2}
\pptPic{.95}{vs-formula.png}\par
{\scriptsize Source: \href{https://radon.readthedocs.io/en/latest/intro.html}{Introduction to Code Metrics}, by Radon\par}
\pptPic{.95}{vs-grades.png}\par
{\scriptsize Source: \href{https://avandeursen.com/2014/08/29/think-twice-before-using-the-maintainability-index/}{Think Twice Before Using the ``Maintainability Index''}, by Arie van Deursen\par}
\par\columnbreak\par
``We decided to be conservative with the thresholds. The desire was that if the index showed red then we would be saying with a high degree of confidence that there was an issue with the code.'' --- \href{https://learn.microsoft.com/en-us/visualstudio/code-quality/code-metrics-maintainability-index-range-and-meaning?view=vs-2022}{Code metrics --- Maintainability index range and meaning} by Microsoft, 2011.
\end{multicols}
}

\qte
  [Rainer Niedermayr]
  {rainer-niedermayr.jpg}
  {We are convinced that Maintainability Index is \ul{nonsense}. We think that it is not sensible to reduce the maintainability of a whole software system to one single indicator.}
  {niedermayr2016}

\plush{\pptThought{``The Maintainability Index does not provide information about the impact on development activities. A value of 57 does not express which maintainability aspects are affected by a bad value.'' --- Rainer Niedermayr}}

\qte
  [Arie van Deursen]
  {arie-van-deursen.jpg}
  {If you are a researcher, \ul{think twice} before using the maintainability index in your experiments. Make sure you study and fully understand the original papers published about it.}
  {deursen2014}

\plush{\pptThought{``Tool smiths and vendors used the exact same formula and coefficients as the 1994 experiments, without any recalibration.'' --- Arie van Deursen}}

\qte
  [Tim Gilboy]
  {tim-gilboy.jpg}
  {If we're going to use the Maintainability Index we should use it to measure \ul{relative} maintainability within our project rather than use it as an \ul{absolute} metric.}
  {gilboy2022}

\plush{\pptThought{\raggedright``Extending the length can significantly decrease Maintainability Index, even if all of the changes cause the code to be clearer and more understandable.'' --- Tim Gilboy}}

\qte
  [Tja\u{s}a Heri\u{c}ko et al.]
  {hericko.png}
  {When comparing maintainability measurements from several Index variants, the perception of maintainability could be impacted by the choice of the Index variant used.}
  {hericko2023}

\pitch{Maintainability Index is supported by a few tools:
\begin{itemize}
\item \href{https://visualstudio.microsoft.com/}{Visual Studio} for C++ and others
\item \href{https://www.sonarsource.com/products/sonarqube/}{SonarQube} for Java
\item \href{https://www.verifysoft.com/en_maintainability.html}{Testwell} for Java and C++
\item \href{https://radon.readthedocs.io/en/latest/index.html}{Radon} for Python
\item \href{https://www.npmjs.com/package/jscomplexity}{jscomplexity} for JavaScript
\item \href{https://github.com/yagipy/maintidx}{maintidx} for Go
\end{itemize}}

\qte
  [Luca Ardito et al.]
  {slr.png}
  {The SLR outcome provided us with \ul{174 software metrics}, among which we identified a set of 15 most commonly mentioned ones, and 19 metric computation tools available to practitioners.}
  {ardito2022}

\pitch{\pptBanner{Milestones of Your Research}
\begin{multicols}{2}
\begin{enumerate}
\item Research Question(s)
\item Research Method
\item Experiments
\item Related Work
\item Results
\item Limitations
\item Discussion
\item Conclusion
\item Introduction
\item Abstract
\item Title
\end{enumerate}\end{multicols}}

\end{document}
