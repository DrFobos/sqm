% (The MIT License)
%
% Copyright (c) 2023 Yegor Bugayenko
%
% Permission is hereby granted, free of charge, to any person obtaining a copy
% of this software and associated documentation files (the 'Software'), to deal
% in the Software without restriction, including without limitation the rights
% to use, copy, modify, merge, publish, distribute, sublicense, and/or sell
% copies of the Software, and to permit persons to whom the Software is
% furnished to do so, subject to the following conditions:
%
% The above copyright notice and this permission notice shall be included in all
% copies or substantial portions of the Software.
%
% THE SOFTWARE IS PROVIDED 'AS IS', WITHOUT WARRANTY OF ANY KIND, EXPRESS OR
% IMPLIED, INCLUDING BUT NOT LIMITED TO THE WARRANTIES OF MERCHANTABILITY,
% FITNESS FOR A PARTICULAR PURPOSE AND NONINFRINGEMENT. IN NO EVENT SHALL THE
% AUTHORS OR COPYRIGHT HOLDERS BE LIABLE FOR ANY CLAIM, DAMAGES OR OTHER
% LIABILITY, WHETHER IN AN ACTION OF CONTRACT, TORT OR OTHERWISE, ARISING FROM,
% OUT OF OR IN CONNECTION WITH THE SOFTWARE OR THE USE OR OTHER DEALINGS IN THE
% SOFTWARE.

\documentclass{article}
\usepackage{../sqm}
\newcommand*\thetitle{Coupling}
\begin{document}

\plush{\sqmTitlePage{6}}

\pitch{\pptQuote{larry-constantine.jpg}{The fewer and simpler the connections between modules, the easier it is to understand each module without reference to other modules.}{Wayne P. Stevens, Glenford J. Myers, and \emph{Larry L. Constantine}, \textit{Structured Design}, IBM Systems Journal 13.2 (1974)}}

\plush{
\pptPic{.8}{coupling.png}\par
{\scriptsize Source: \url{https://www.geeksforgeeks.org/coupling-in-java/}\par}}

\plush{
\pptPic{.8}{coupling-2.png}\par
{\scriptsize Source: \url{https://www.javatpoint.com/software-engineering-coupling-and-cohesion}\par}}

\pitch{\pptQuote{glenford-myers.jpg}{Coupling is the measure of the strength of association established by a connection from one module to another. Strong coupling \emph{complicates} a system since a module is harder to \emph{understand}, \emph{change}, or \emph{correct} by itself if it is highly interrelated with other modules. Complexity can be reduced by designing systems with the weakest possible coupling between modules.}{Wayne P. Stevens, \emph{Glenford J. Myers}, and Larry L. Constantine, \textit{Structured Design}, IBM Systems Journal 13.2 (1974)}}

\pitch{\pptQuote{wayne-stevens.jpg}{The degree of coupling established by a particular connection is a function of several factors, and thus it is \emph{difficult to establish} a simple index of coupling. Coupling depends (1)~on how complicated the connection is, (2)~on whether the connection refers to the module itself or something inside it, and (3)~on what is being sent or received.}{\emph{Wayne P. Stevens}, Glenford J. Myers, and Larry L. Constantine, \textit{Structured Design}, IBM Systems Journal 13.2 (1974)}}

\pitch{\pptQuote{cbo.png}{\emph{Coupling Between Objects (CBO)} --- for a class is a count of the number of other classes to which it is coupled.}{Shyam R. Chidamber and Chris F. Kemerer, \textit{A metrics suite for object oriented design}, IEEE Transactions on Software Engineering, 20.6, 1994}}

\pitch{\pptQuote{dcc.png}{\emph{Direct Class Coupling (DCC)} --- this metric is a count of the different number of classes that a class is directly related to. The metric includes classes that are directly related by attribute declarations and message passing (parameters) in methods.}{Jagdish Bansiya and Carl G. Davis, \textit{A Hierarchical Model for Object-Oriented Design Quality Assessment}, IEEE Transactions on Software Engineering, 28.1, 2002}}

\pitch{\pptQuote{martin-fowler.jpg}{The biggest problems come from uncontrolled coupling at the \emph{upper levels}. I don't worry about the number of modules coupled together, but I look at the pattern of dependency relationship between the modules.}{Martin Fowler, \textit{Reducing Coupling}, IEEE Software, 2001}}

\pitch{\pptQuote{steve-mcconnell.jpg}{Low-to-medium \emph{fan-out} means having a given class use a low-to-medium number of other classes. High fan-out (more than about seven) indicates that a class uses a large number of other classes and may therefore be overly complex. High \emph{fan-in} refers to having a high number of classes that use a given class. High fan-in implies that a system has been designed to make good use of utility classes at the lower levels in the system.}{Steven McConnell, \textit{Code Complete}, 2004}}

\plush{
\pptPic{.95}{kokash.png}\par
{\scriptsize (c) Natalia Kokash, Leiden Institute of Advanced Computer Science\par}}

\pitch{\pptQuote{study.png}{We also found evidence of certain `key' classes (with both high fan-in and fan-out) and `client' and `server'-type classes with just high fan-out and fan-in, respectively.}{A. Mubarak et al., \textit{An evolutionary study of fan-in and fan-out metrics in OSS}, Proceedings of the 4th International Conference on Research Challenges in Information Science (RCIS), 2010}}

\pitch{Fan-out, as a metric, is supported by a few tools:
\begin{itemize}
\item \href{https://checkstyle.sourceforge.io/apidocs/com/puppycrawl/tools/checkstyle/checks/metrics/ClassFanOutComplexityCheck.html}{Checkstyle} for Java
\item \href{https://github.com/sarnold/cccc}{CCCC} for C++, C, and Java
\item \href{https://pypi.org/project/module-coupling-metrics/}{module-coupling-metrics} for Python
\end{itemize}}

\pptBanner{Fear of Decoupling}
\begin{multicols}{2}
{\scriptsize\begin{ffcode}
class Temperature {
  private int t;
  public String toString() {
    return String.format("%d F", this.t);
  }
}

Temperature x = new Temperature();
String txt = x.toString();
String[] parts = txt.split(" ");
int t = Integer.parseInt(parts[0]);
\end{ffcode}
}
\par\columnbreak\par
``The larger the number (or the mean of all numbers), the worse the design: in good design we are not supposed to take something out of a method and then do some complex processing. The \emph{distance metric} will tell us exactly that: how many times, and by how much, we violated the principle of loose coupling.''\par
{\scriptsize \url{https://www.yegor256.com/2020/10/27/distance-of-coupling.html}\par}
\end{multicols}
\plush{}

\pptBanner{Temporal Coupling}
\begin{multicols}{2}
{\small\begin{ffcode}
List<String> list =
  new LinkedList();
Foo.append(list, "Jeff");
Foo.append(list, "Walter");
return list;
\end{ffcode}
}
\par\columnbreak\par
{\small\begin{ffcode}
return Foo.with(
  Foo.with(
    new LinkedList(),
    "Jeff"
  ),
  "Walter"
);
\end{ffcode}
}
\end{multicols}\par
{\scriptsize \url{https://www.yegor256.com/2015/12/08/temporal-coupling-between-method-calls.html}\par}
\plush{}

\pptBanner{Distance of Coupling}
\begin{multicols}{2}
{\scriptsize\begin{ffcode}
class Temperature {
  private int t;
  public String toString() {
    return String.format("%d F", this.t);
  }
}

Temperature x = new Temperature();
String txt = x.toString();
String[] parts = txt.split(" ");
int t = Integer.parseInt(parts[0]);
\end{ffcode}
}
\par\columnbreak\par
``The larger the number (or the mean of all numbers), the worse the design: in good design we are not supposed to take something out of a method and then do some complex processing. The \emph{distance metric} will tell us exactly that: how many times, and by how much, we violated the principle of loose coupling.''\par
{\scriptsize \url{https://www.yegor256.com/2020/10/27/distance-of-coupling.html}\par}
\end{multicols}
\plush{}

\plush{
  \pptBanner{Read this:}\par
  \textit{Structured Design}, Wayne P. Stevens, et al., IBM Systems Journal, 13.2, 1974\par
  \textit{A Hierarchical Model for Object-Oriented Design Quality Assessment}, Jagdish Bansiya and Carl G. Davis, IEEE Transactions on Software Engineering, 28.1, 2022\par
  \textit{An Overview of Various Object Oriented Metrics}, Brij Mohan Goel and Pradeep Kumar Bhatia, International Journal of Information Technology \& Systems, 2.1, 2014\par
  \href{https://www.yegor256.com/2020/10/27/distance-of-coupling.html}{New Metric: the Distance of Coupling} (2020)\par
  \href{https://www.yegor256.com/2018/09/18/fear-of-coupling.html}{Fear of Decoupling} (2018)\par
  \href{https://www.yegor256.com/2022/06/05/reflection-means-hidden-coupling.html}{Reflection Means Hidden Coupling} (2022)\par
}

\end{document}
