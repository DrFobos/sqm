% (The MIT License)
%
% Copyright (c) 2023-2024 Yegor Bugayenko
%
% Permission is hereby granted, free of charge, to any person obtaining a copy
% of this software and associated documentation files (the 'Software'), to deal
% in the Software without restriction, including without limitation the rights
% to use, copy, modify, merge, publish, distribute, sublicense, and/or sell
% copies of the Software, and to permit persons to whom the Software is
% furnished to do so, subject to the following conditions:
%
% The above copyright notice and this permission notice shall be included in all
% copies or substantial portions of the Software.
%
% THE SOFTWARE IS PROVIDED 'AS IS', WITHOUT WARRANTY OF ANY KIND, EXPRESS OR
% IMPLIED, INCLUDING BUT NOT LIMITED TO THE WARRANTIES OF MERCHANTABILITY,
% FITNESS FOR A PARTICULAR PURPOSE AND NONINFRINGEMENT. IN NO EVENT SHALL THE
% AUTHORS OR COPYRIGHT HOLDERS BE LIABLE FOR ANY CLAIM, DAMAGES OR OTHER
% LIABILITY, WHETHER IN AN ACTION OF CONTRACT, TORT OR OTHERWISE, ARISING FROM,
% OUT OF OR IN CONNECTION WITH THE SOFTWARE OR THE USE OR OTHER DEALINGS IN THE
% SOFTWARE.

\documentclass[nobrand,anonymous,nodate,nosecurity]{huawei}
\usepackage[utf8]{inputenc}
\usepackage[T2A]{fontenc}
\usepackage[russian,english]{babel}
\usepackage{booktabs}
\usepackage{tabularx}
\usepackage{multicol}
\usepackage{href-ul}
\usepackage{ffcode}
\usepackage{soul}
\begin{document}

\newcommand\head[2]{#1: & #2 \\}
\begin{tabularx}{\textwidth}{r>{\raggedright\arraybackslash}X}
\head{Название}
  {``Практические методы повышения качества программного кода''}
\head{Департамент}
  {Кафедра Программной Инженерии (КПИ)}
\head{Период реализации}
  {1,2,3 модули 2023/2024}
\head{Язык}
  {Русский}
\head{Видео}
  {\href{https://www.youtube.com/playlist?list=PLaIsQH4uc08xyXRhhYPHh-Yam2kEwNaLl}{YouTube}}
\head{Слайды}
  {\href{https://github.com/yegor256/sqm}{GitHub}}
\head{Охват аудитории}
  {Для всего кампуса}
\head{Объем дисциплины}
  {48 часов семинары + 66 часов самостоятельная работа}
\head{Онлайн курс}
  {---}
\head{Технологии реализации}
  {Лекции: офлайн-занятия, практика: офлайн-занятия}
\head{Разработчики}
  {Бугаенко Егор Георгиевич}
\head{Утверждение}
  {---}
\end{tabularx}

\section{Цели курса}

В лекционной части курса рассматриваются инструменты оценки качества кода и методы его повышения. Теоретическое изложение не привязано к конкретным языкам программирования. Практическая часть курса предполагает самостоятельную работу студентов над анализом крупных open source проектов с целью выявления закономерностей и проблем с качеством.

Ожидается, что по окончанию курса студент будет:
\begin{itemize}
    \item Понимать, из чего состоит оценка качества программных продуктов;
    \item Уметь проводить научные исследования в сфере анализа репозиториев с программным кодом;
    \item Уметь пользоваться инструментами анализа качества кода.
\end{itemize}

Предполагается, что к началу курса студент умеет:
\begin{itemize}
    \item Самостоятельно программировать на одном из популярных языков, таких как Java, JavaScript, Python или C++;
    \item Самостоятельно управлять репозиторием проекта с помощью Git;
    \item Самостоятельно моделировать архитектуру программного проекта, используя UML.
\end{itemize}

Лучшие студенты опубликуют результаты своего исследования на одной из научных конференций.

\section{Лекции}

Всего в курсе 24 лекции по два академических часа (80 минут каждая лекция):
\begin{enumerate}
    \setlength\itemsep{0em}
    \item Lines of Code
    \item McCabe's Cyclomatic Complexity
    \item Cognitive Complexity
    \item Halstead Complexity
    \item Maintainability Index

    \item Coupling
    \item LCOM 1, 2, 3, 4, 5
    \item TCC and LCC
    \item CAMC and NHD
    \item Object Dimensions

    \item Clone Coverage
    \item Dead Code
    \item Code Churn
    \item Tech Debt

    \item Code Coverage
    \item Mutation Coverage

    \item Function Points

    \item Defects Density
    \item Comments Density
    \item Commits Density

    \item Builds
    \item Code Style
    \item Static Analysis

    \item Neural Metrics
\end{enumerate}

Порядок лекций может меняться, а тематика корректироваться в зависимости от интереса аудитории.

Тип аудиторий: лекционные (для лекционных занятий).

Оснащение аудиторий: наличие Wi-Fi.

\section{Аттестация}

В рамках курса студенты должны разделиться на группы по 1--3 человека. Каждая группа должна выбрать одну из исследовательских тем, предложенных на первой лекции. По выбранной теме необходимо провести исследование и оформить его результаты в научную статью на английском языке, объемом 7--12 страниц формата \href{https://ctan.org/pkg/acmart}{acmart/sigplan}. Статья должна содержать обязательные разделы: Abstract, Introduction, Related Work, Method, Experimental Results, Discussion, Conclusion и References.

Студент может также набрать баллы путем написания кода в одном из open source репозитории.

Оценка за курс складывается из суммы баллов:

\renewcommand{\arraystretch}{1}
\begin{tabularx}{\textwidth}{>{\raggedright}p{4in}>{\raggedleft\arraybackslash}X}
\toprule
Результат & Баллы \\
\midrule
Студент представил черновик статьи (2+ страниц) до конца первого модуля & 1 \\
Студент представил черновик статьи (4+ страниц) до конца второго модуля & 1 \\
Студент успешно отправил 4+ pull requests & 1 \\
Студент успешно отправил 8+ pull requests & 2 \\
Студент успешно отправил 12+ pull requests & 4 \\
Студент посетил 10 или более лекций & 1 \\
Студент посетил 20 или более лекций & 3 \\
Студент отправил статью на конференцию после разрешения преподавателя & 1 \\
Статья принята на конференции CORE-B & 7 \\
Статья принята на конференции CORE-A & 10 \\
\bottomrule
\end{tabularx}

Пересдача невозможна.

\section{Самостоятельная работа}

Работая самостоятельно над темой исследования, ожидается, что студент выполнит следующее:
\begin{itemize}
    \item Изучит существующие статьи и книги, касающиеся темы исследования;
    \item Сформулирует research question;
    \item Определит метод исследования;
    \item Проведет исследования и соберет данные;
    \item Проанализирует результаты;
    \item Выявит слабые стороны и опишет их;
    \item Сделает вывод;
    \item Оформит статью.
\end{itemize}

Дополнительную помощь можно получить в этой статье:
\href{https://www.yegor256.com/2022/08/24/research-paper-template.html}{Research Paper Simple Template}
(и в материалах, на которые статья ссылается).

\section{Литература}

\begin{multicols}{2}\small\raggedright
Len Bass et al., \ul{Software Architecture in Practice}\\[3pt]
Paul Clements et al., \ul{Documenting Software Architectures: Views and Beyond}\\[3pt]
\nospell{Karl Wiegers} et al., \ul{Software Requirements}\\[3pt]
{\nospell{Alistair Cockburn}}, \ul{Writing Effective Use Cases}\\[3pt]
{\nospell{Steve McConnell}}, \ul{Software Estimation: Demystifying the Black Art}\\[3pt]
{Robert Martin}, \ul{Clean Architecture: A Craftsman's Guide to Software Structure and Design}\\[3pt]
{Steve McConnell}, \ul{Code Complete}\\[3pt]
{Frederick Brooks Jr.}, \ul{Mythical Man-Month, The: Essays on Software Engineering}\\[3pt]
{David Thomas et al.}, \ul{The Pragmatic Programmer: Your Journey To Mastery}\\[3pt]
{Robert C. Martin}, \ul{Clean Code: A Handbook of Agile Software Craftsmanship}\\[3pt]
{\nospell{Grady Booch} et al.}, \ul{Object-Oriented Analysis and Design with Applications}\\[3pt]
{\nospell{Bjarne Stroustrup}}, \ul{Programming: Principles and Practice Using C++}\\[3pt]
{\nospell{Brett McLaughlin} et al.}, \ul{Head First Object-Oriented Analysis and Design: A Brain Friendly Guide to OOA\&D}\\[3pt]
{David West}, \ul{Object Thinking}\\[3pt]
{Eric Evans}, \ul{Domain-Driven Design: Tackling Complexity in the Heart of Software}\\[3pt]
{Yegor Bugayenko}, \ul{Code Ahead}\\[3pt]
{Yegor Bugayenko}, \ul{Elegant Objects}\\[3pt]
{Michael Feathers}, \ul{Working Effectively with Legacy Code}\\[3pt]
{Martin Fowler}, \ul{Refactoring: Improving the Design of Existing Code}\\[3pt]
{Erich Gamma et al.}, \ul{Design Patterns: Elements of Reusable Object-Oriented Software}\\[3pt]
{Martin Fowler}, \ul{UML Distilled}\\[3pt]
{\nospell{Anneke Kleppe} et al.}, \ul{MDA Explained: The Model Driven Architecture: Practice and Promise}\\[3pt]
{\nospell{Jez Humble} et al.}, \ul{Continuous Delivery: Reliable Software Releases through Build, Test, and Deployment Automation}\\[3pt]
{\nospell{Michael T. Nygard}}, \ul{Release It!: Design and Deploy Production-Ready Software}\\[3pt]
\end{multicols}

\end{document}
