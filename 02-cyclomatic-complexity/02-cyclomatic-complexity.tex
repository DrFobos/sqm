% (The MIT License)
%
% Copyright (c) 2023-2024 Yegor Bugayenko
%
% Permission is hereby granted, free of charge, to any person obtaining a copy
% of this software and associated documentation files (the 'Software'), to deal
% in the Software without restriction, including without limitation the rights
% to use, copy, modify, merge, publish, distribute, sublicense, and/or sell
% copies of the Software, and to permit persons to whom the Software is
% furnished to do so, subject to the following conditions:
%
% The above copyright notice and this permission notice shall be included in all
% copies or substantial portions of the Software.
%
% THE SOFTWARE IS PROVIDED 'AS IS', WITHOUT WARRANTY OF ANY KIND, EXPRESS OR
% IMPLIED, INCLUDING BUT NOT LIMITED TO THE WARRANTIES OF MERCHANTABILITY,
% FITNESS FOR A PARTICULAR PURPOSE AND NONINFRINGEMENT. IN NO EVENT SHALL THE
% AUTHORS OR COPYRIGHT HOLDERS BE LIABLE FOR ANY CLAIM, DAMAGES OR OTHER
% LIABILITY, WHETHER IN AN ACTION OF CONTRACT, TORT OR OTHERWISE, ARISING FROM,
% OUT OF OR IN CONNECTION WITH THE SOFTWARE OR THE USE OR OTHER DEALINGS IN THE
% SOFTWARE.

\documentclass{article}
\usepackage{../sqm}
\newcommand*\thetitle{Cyclomatic Complexity}
\begin{document}

\plush{\sqmTitlePage{2}}

\thought{Some programmers mistakenly feel proud of higher complexity of their code.}

\pitch{\pptQuote{thomas-mccabe.jpg}{Cyclomatic Complexity (CC) is a count of the number of decisions in the source code. The higher the count, the more complex the code. The formula is simple: \(C=E-N+2\).}{Thomas~J.~McCabe,~Sr, 1976}}

\plush{
\pptBanner{What is the complexity of this program?}
\pptPic{.5}{cc.png}\par
I found this picture \href{https://craftofcoding.wordpress.com/2017/06/18/coding-a-small-note-on-cyclomatic-complexity/}{here}.}

\thought{In his presentation "\textit{Software Quality Metrics to Identify Risk}", Tom McCabe introduces the following categorisation to interpret cyclomatic complexity: 1--10 little risk, 11--20 moderate risk, 21--50 high risk, 50+ very high risk.}

\thought{Graylin JAY et al., \textit{Cyclomatic Complexity and Lines of Code: Empirical Evidence of a Stable Linear Relationship}, Journal of Software Engineering \& Applications, Vol. 2, 2009, pp. 137--143}

\thought{What is a complexity of a class? How about a module?}

\thought{\emph{Feature creep} is one of the most common sources of cost and schedule overruns; it can even kill products and projects --- Wikipedia.}

\thought{Tom McCabe suggested to prohibit functions where CC is larger than \emph{ten}. Modern static analyzers may help you control this automatically.}

\plush{
  \pptBanner{Read this:}\par
  \href{https://www.yegor256.com/2015/06/29/simple-diagrams.html}{The Better Architect You Are, The Simpler Your Diagrams} (2015)\par
  \href{https://www.yegor256.com/2014/10/26/hacker-vs-programmer-mentality.html}{Are You a Hacker or a Designer?} (2014)\par
}

\end{document}
